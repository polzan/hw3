\documentclass[a4paper,oneside]{article}
\usepackage[T1]{fontenc}
\usepackage[utf8]{inputenc}
\usepackage[english]{babel}

\usepackage[margin=2.54cm]{geometry}
\usepackage{amsmath}
\usepackage{siunitx}
\usepackage{listings}
\usepackage{color}
\usepackage{textcomp}
\usepackage{graphicx}
\usepackage{xr}
\usepackage{subcaption}
%\usepackage{changepage}
\usepackage[section]{placeins}
%\usepackage{hyperref}

%\strictpagecheck
\externaldocument{hw3_code}

\definecolor{matlabgreen}{RGB}{28,172,0}
\definecolor{matlablilas}{RGB}{170,55,241}

\newcommand{\includecode}[1]{\lstinputlisting[caption={\ttfamily #1.m},label={lst:#1}]{matlab/#1.m}}
\newcommand{\inlinecode}[1]{\lstinline[basicstyle=\ttfamily,keywordstyle={},stringstyle={},commentstyle={\itshape}]{#1}}

\renewcommand{\vec}[1]{\underline{#1}}
\renewcommand{\Re}[1]{\operatorname{Re}\left[#1\right]}
\newcommand{\E}[1]{\operatorname{E}\left[#1\right]}
\newcommand{\norm}[1]{\left\lVert#1\right\rVert}
\newcommand{\abs}[1]{\left|#1\right|}
\newcommand{\F}[1]{\operatorname{\mathcal{F}}\left[#1\right]}
\newcommand{\ceil}[1]{\left\lceil#1\right\rceil}
\newcommand{\floor}[1]{\left\lfloor#1\right\rfloor}
\newcommand{\Prob}[1]{\operatorname{P}\left[#1\right]}
\newcommand{\ProbC}[2]{\operatorname{P}\left[#1\middle|#2\right]}
\newcommand{\ind}[1]{\operatorname{\mathbbm{1}}\left\{#1\right\}}
\DeclareMathOperator*{\argmin}{argmin}
\DeclareMathOperator*{\argmax}{argmax}
\newcommand{\distr}[0]{\sim}
\newcommand{\unif}[1]{\mathcal{U}_{#1}}

\newcommand{\vsigma}[0]{\vec{\sigma}}

\newcolumntype{L}{>{$}l<{$}}
\newcolumntype{R}{>{$}r<{$}}
\newcolumntype{C}{>{$}c<{$}}

\author{Enrico Polo \and Riccardo Zanol}
\title{Homework 3}

\begin{document}
\lstset{
  language=Matlab,
  basicstyle={\ttfamily \footnotesize},
  breaklines=true,
  morekeywords={true,false,warning,xlim,ylim},
  keywordstyle=\color{blue},
  stringstyle=\color{matlablilas},
  commentstyle={\color{matlabgreen} \itshape},
  numberstyle={\ttfamily \tiny},
  frame=leftline,
  showstringspaces=false,
  numbers=left,
  upquote=true,
}
\maketitle
\section{Transmitter}
In the transmitter we generate the sequence of symbols $a_k$ by
producing a uniformly distributed sequnce of bits twice as long and
mapping each pair of bits to a symbol of the QPSK constellation $
\mathcal{A} = \{(1+j),(1-j),(-1-j),(-1+j)\}$. Then we upsample $a_k$
by a factor of 4 and filter it using the transfer function
\begin{equation}
  Q_c(z) = \frac{\beta z^{-10}}{1 - \alpha z^{-1}}
\end{equation}
that models the series-combination of the modulator filter and
  the channel.  Since we will need it to have a finite length,
because the matched filter would otherwise not be causal, we truncate
its impulse response at $n=33$ (when $n \geq 34$, $q_c(nT/4) \leq
5\cdot10^{-5}$). In Fig.~\ref{plot:qc} we plot $q_c$ and in
Fig.~\ref{plot:Qf} there is the corresponding frequency response.
\begin{figure}[htbp]
  \centering
  \includegraphics[width=0.6\textwidth]{matlab/plot_qc}
  \caption{Impulse response of the combination of the modulator and
    the channel}
  \label{plot:qc}
\end{figure}
\begin{figure}[htbp]
  \centering
  \includegraphics[width=0.6\textwidth]{matlab/plot_Qf}
  \caption{Frequency response of the combination of the modulator and
    the channel}
  \label{plot:Qf}
\end{figure}

We then add the noise $w_c(nT/4)$, assumed to be
a complex Gaussian with power spectral density $\mathcal{P}_{w_c}(f) =
N_0$ in the band of the signal $S_c(nT/4)$. The noise power and PSD
can be obtained from the SNR:
\begin{align}
  \Gamma &= \frac{\sigma^2_a E_{q_c}}{\sigma^2_{w_c}} \\
  N_0 &= \sigma^2_{w_c}\frac{T}{4}
\end{align}
where $E_{q_c}$ is the energy of the filter $q_c$ and the power of the
symbol sequence is $\sigma^2_a = 2$.

The signal that gets to each one of the following receivers is
$r_c(nT/4) = S_c(nT/4) + w_c(nT/4)$.

\section{Matched filters}
\label{sec:mf}
To implement the receivers scheme a) and b), knowing the equivalent trasmitter impulse response $q_c$, we firstly designed the $g_m$ filter by usign the relation, 
\begin{align}
  g_m(n T/4) &= q_c^* (\hat{t_0}-n T/4) & \hat{t_0} = 33 T/4&
\end{align}

\begin{figure}[htbp]
  \centering
  \includegraphics[width=0.6\textwidth]{matlab/plot_gm}
  \caption{Match filter impulse response}
  \label{plot:gm}
\end{figure}

where we choose the smallest value we can of $\hat{t_0}$ that gives us a causal filter. 
Then, by inspecting the overall impulse response before the sampler $q_r(n T/4) = (q_c * g_m)(n T/4)$, we find a proper value of $t_0$ that in this case is equal to $33T/4$ ($\Bar{t_0} = 33$). With this choice we are sampling the riceived signal $x(nT/4) = (r_c*gm)(nT/4)$ in the peak of $q_r$ in order to reach the best performances.
Now we discuss the two configuration separately.
\subsection{Linear Equalizer}
\label{sec:le}
To implement the Linear Equalizer we estmate the optimum coefficients of the filter $c$ through the relation:
\begin{align}
\vec{c}_{opt}& = R^{-1} \vec{p}&
\label{eq:c_opt}  
\end{align} 
in which 
\begin{align}
&\vec{p} = \sigma_a^2 h_{D-p}^*&
\label{eq:p}  
\end{align}
where $h$ is the overall impulse response $q_r$ sampled at $T$ and $\sigma_a^2$ is the power of the transmitted simbols. For the matrix $R$ the following relation holds in this case:
\begin{align}
 R_{p,q} &= \sigma^2_a \sum_{j=-N_1}^{N_2}h_jh^*_{j-(p-q)} + r_{\tilde{w}}(p-q) & p,q = 0,1,.....,M1-1&
  \label{eq:R_le}
\end{align}
in which $r_{\tilde{w}}(k) = N_0 r_{g_M}(k)$; with $r_{g_M}(k)$ the deterministic autocorrelation of $g_m$.
The two parameters in this configuration are the delay D and the length of the  equalizer M1. By trying some configurations we find that a suitable value for D is 1 and for M1 is 3. If we increase more the value of M1 we don't have a significant improvement in the performances, so we decided to stop here. For $D\neq 1$ indeed the sistem starts to work significantly worse. With this settings we find the behaviour of the overall system impulse response $\psi(p) = \sum_{l=0}^{M1-1} c_l h_{p-l}$ showed in Fig.~\ref{plot:psi_le}. 

\begin{figure}[h]
  \centering
  \includegraphics[width=0.6\textwidth]{matlab/plot_psi_le}
  \caption{LE overall impulse response}
  \label{plot:psi_le}
\end{figure}

In conclusion we have set: D = 1, M1 = 3, $\Bar{t_0} = 33$. The coefficents of the filter $c$ are reported in Tab.~\ref{tab:c_le}.

\begin{table}
  \centering
  \begin{tabular}{LL}
    c_0 & -0,2016 \\
    c_1 & 1,0601  \\
    c_2 & -0,2016
  \end{tabular}
  \caption{Coefficietns of the filter $c$ of the LE}
  \label{tab:c_le}
\end{table}

\subsection{Decision Feedback Equalizer}
\label{sec:dfe}
To implement the filter $c$ of the Decision Filter Equalizer we used 
the same approach of the previous point (Equations (\ref{eq:c_opt}), (\ref{eq:p}) and (\ref{eq:R_le})). The only difference is 
the computation of R that it's done using
\begin{align}
 R_{p,q} &= \sigma^2_a\left( \sum_{j=-N_1}^{N_2}h_jh^*_{j-(p-q)} - \sum_{j=1}^{M_2}h_{j+D-q}h^*_{j+D-p} \right) + r_{\tilde{w}}(p-q)
\end{align}
in which M2 is the length of the feedback filter $b$ and the others quantities are the same as before.
The coefficients of the last filter are finally estimate using the relation 
\begin{align}
b_i &= - \psi_{D+i}&  i &= 1,2,....,M2
\end{align}
Also in this case we have choose a value of D = 1 but now the first filter can be simpler because the postcursus are equilized by $b$. In fact, trying the various configuration we have found that M1 = 3 and M2 = 2 it's enough to our purposes. If we increase any of the two quantities the performance are quite the same, but, onthe other hand, if we decrease M1 or M2 the error probability starts to increase. We report the behaviour of $\psi$ in Fig. \ref{plot:psi_dfe_1} and the final configuration we used:
M1 = 3, M2 = 2, D = 1, $\Bar{t_0} = 33$. The filters $\vec{c}_{opt}$ and $\vec{b}$ are reported in Tab.~\ref{tab:c_dfe}~and~\ref{tab:b_dfe}. The match filter and the configuration of the sampler are the same of point a).

\begin{table}
  \centering
\begin{tabular}{LL}
  c_0 & -0.2017 \\
  c_1 & 1.0234 \\
  c_2 & -0.0158
\end{tabular}
\caption{Filter $c$ used in the DFE}
\label{tab:c_dfe}
\end{table}
\begin{table}
  \centering
\begin{tabular}{LL}
  b_1 & -0.1823 \\
  b_2 & -0.0367
\end{tabular}
\caption{Filter $b$ used in the DFE}
\label{tab:b_dfe}
\end{table}

\begin{figure}[h]
  \centering
  \includegraphics[width=0.6\textwidth]{matlab/plot_psi_dfe}
  \caption{DFE $\psi(kT)$ behaviour}
  \label{plot:psi_dfe_1}
\end{figure}



\section{Anti-aliasing filters}

\subsection{Sampling with period $T/2$}
For the configuration c) we have usa a lowpass linear FIR filter of order 20 to build $gaa$, whose frequency response and impulse response are shown in Fig. \ref{plot:Gaa2} and Fig. \ref{plot:gaa2} respectively. 
\begin{figure}[h]
  \centering
  \includegraphics[width=0.6\textwidth]{matlab/plot_gaa_freq}
  \caption{Frequency response of $g_{aa}$ in dB}
  \label{plot:Gaa2}
\end{figure}
\begin{figure}[h]
  \centering
  \includegraphics[width=0.6\textwidth]{matlab/plot_gaa}
  \caption{$g_{aa}$ impulse response}
  \label{plot:gaa2}
\end{figure}

 

\begin{figure}[h]
  \centering
  \includegraphics[width=0.6\textwidth]{matlab/plot_psi_aa}
  \caption{Sampled $\psi(kT)$ behaviour}
  \label{plot:psi_aa}
\end{figure}
Proceeding like in the previous points we inspected the impulse response before the sampler $q_r(n T/4) = (q_c * g_m)(n T/4)$ and we find that, to sample in the peak, we must choose $\Bar{t_0} = 20$, but in this case we sample every T/2. The design of the optimum filter $c$ it's done as in case b), through the equation (8), then we downsample the received signal by a factor $F_0 = 2$ to return at time T. Finally to build the $b$ filter we used the relation $b_i = - \psi_{2(D+i)}$ that takes into account the fact that the filter $c$ works at T/2 but after the downsampler we have a sample (and a postcursus of psi that needs to be cancelled) every T seconds. In our tries we find the following settings as the best: M1 = 3, M2 = 2, D = 1, $\Bar{t_0} = 20$ and we obtained: $\vec{c}_{opt} = [0.0012, 0.0451, 0.2879]^T$ $\vec{b} = [-0.0518, -0.0086]^T$. The correspondent $\psi$ at time T is showed in Fig. \ref{plot:psi_aa}.

\subsection{Sampling with period $T$}
To implement setting d) we find that using an order 10, linear FIR, lowpass filter as $g_{aa}$ we can get good performances. Also in this case the frequency response and impulse response are showed in Fig. \ref{plot:gaa_freq_T} and Fig. \ref{plot:gaa}. using a different size for the anti-aliasing filter we need to recompute the optimal value of $\Bar{t_0}$ and we find that it must be equal to 16. This configuration is similar to the previous one, but we don't have to account for the different timing between filters $c$ and $b$. So, as usual we used eq. (8) to estimate the $\vec{c}_{opt}$ and then we find $b_i = -\psi_{D+i}$. Making some tests with this setup we fiin the following results: M1 = 2, D = 1, M2 = 3, $\Bar{t_0} = 16$, $\vec{c}_{opt} = [0.0013, 0.0914]^T$, $\vec{b} = [-0.0095, -0.0019, -0.0004]^T$ . The behaviour of $\psi$ is showed in Fig. \ref{plot:gaa_psi_T}.
\begin{figure}[h]
  \centering
  \includegraphics[width=0.6\textwidth]{matlab/plot_gaa_freq_T}
  \caption{Frequency response of $g_{aa}$ in dB}
  \label{plot:gaa_freq_T}
\end{figure}

\begin{figure}[h]
  \centering
  \includegraphics[width=0.6\textwidth]{matlab/plot_gaa_T}
  \caption{$g_{aa}$ impulse response}
  \label{plot:gaa}
\end{figure}

\begin{figure}[h]
  \centering
  \includegraphics[width=0.6\textwidth]{matlab/plot_psi_aa_T}
  \caption{$\psi(kT)$ behaviour}
  \label{plot:gaa_psi_T}
\end{figure}

\section{Viterbi}
\label{sec:viterbi}
In the Viterbi receiver we use the same filter $g_M(nT/4) = q_c^*((\overline{t_0} - n)T/4)$ of
section~\ref{sec:mf}, matched to the response of
the transmitter and the channel $q_c$, to maximize the
  SNR before the equalizer and we also sample in the same way
starting at $t_0 = \overline{t_0}\frac{T}{4} = 33\frac{T}{4}$ with
period $T$.

Before applying the Viterbi algorithm we design a filter $c$ with the
method based on the Wiener filter already used in
section~\ref{sec:dfe}:
\begin{align}
  \vec{p}_p &= \sigma^2_ah^*_{D-p} \\
  R_{p,q} &= \sigma^2_a\left( \sum_{j=-N_1}^{N_2}h_jh^*_{j-(p-q)} - \sum_{j=1}^{M_2}h_{j+D-q}h^*_{j+D-p} \right) + r_{\tilde{w}}(p-q) \\
    \vec{c} &= R^{-1} \vec{p}
\end{align}
where $h = q_c * g_M$ and $r_{\tilde{w}}(n) = N_0 r_{g_M}(nT)$. The
parameters that we choose are the same as in section \ref{sec:dfe}:
$M1 = 3$, $M2 = 2$, $D=1$ and the filter coefficients are the same ones of
Tab.~\ref{tab:c_dfe}.

The overall system response before the Viterbi detector is $\psi(kT) =
(q_c * g_M * c)_{(kT)}$ and it is shown in Fig.~\ref{plot:psi_dfe}.
\begin{figure}[htbp]
  \centering
  \includegraphics[width=0.7\textwidth]{matlab/plot_psi_dfe}
  \caption{Overall system response $\psi(kT)$}
  \label{plot:psi_dfe}
\end{figure}
We can see that the filter $c$ has eliminated the effect of the
precursors, so the Viterbi algorithm only has to take into account the
two non-zero postcursors of $\eta_n = \psi((n+D)T)$ and its parameters
for the state will be $L_1 = 0$, $L_2 = M_2 = 2$.

To implement the Viterbi algorithm we first define the states
\begin{equation}
  \vec{s}_k = (a_k,a_{k-1}) \in \mathcal{S} = \{ \vec{\sigma}_1, \vec{\sigma}_2, \dots \vec{\sigma}_{16} \}
\end{equation}
that can be any combination of two symbols taken from the
constellation $\mathcal{A}$, then we pre-compute:
\begin{itemize}
  \item if it is possible to have a transition from a state
    $\vec{s}_{k-1} = \vec{\sigma}_i$ to the successive state
    $\vec{s}_{k} = \vec{\sigma}_j$, for every pair of states
    $\vec{\sigma}_j$, $\vec{\sigma}_i$    
\item the value of the sample we would receive if
  $(\vec{s}_k,\vec{s}_{k-1})$ were equal to $(\vsigma_j, \vsigma_i)$
  \begin{equation}
    u_k = f(\vec{\sigma}_j, \vec{\sigma}_i) =
    \sum_{n=-L_1}^{L_2}\psi_{n+D}a_{k-n}
  \end{equation} wherever a transition from $\vec{\sigma}_i$ to $\vec{\sigma}_j$ is
  possible.
\end{itemize}
We then initialize the path metric values at state $k=-1$
\begin{equation}
  \Gamma(\vec{s}_{k-1} = \vec{\sigma}_j) = 0 \qquad j=1,2,\dots16
\end{equation}
because we have no information on the state of the system, so each one
is equally likely, and start to process the input signal $\rho_k$
sample by sample following the Viterbi algorithm. At every iteration
we compute for each state $\vsigma_j$:
\begin{itemize}
\item the branch metrics
  \begin{equation}
    \abs{\rho_k - u_k}^2 \qquad \forall u_k = f(\vsigma_j,\vsigma_i)
  \end{equation}
    where it is possible to move from $\vsigma_i$ to $\vsigma_j$,
  \item the corresponding path metrics
    \begin{equation}
      \Gamma(\vec{s}_{k-1} = \vec{\sigma}_i) + \abs{\rho_k - u_k}^2 \qquad \forall \vsigma_i ,
    \end{equation}
  \item the path metric that we assign to state $\vsigma_j$ in the
    current iteration $k$
    \begin{equation}
      \Gamma(\vec{s}_k = \vec{\sigma}_j) = \min_{\vsigma_i} \left[ \Gamma(\vec{s}_{k-1} = \vec{\sigma}_i) + \abs{\rho_k - u_k}^2 \right]
    \end{equation}
    and we also store the state that minimizes the path metric,
    ${\vsigma_i}_{opt}$, in the survivor sequence
    \begin{equation}
      \mathcal{L}(\vec{s}_k = \vsigma_j) = (\mathcal{L}(\vec{s}_{k-1} = {\vsigma_i}_{opt}), \vsigma_j) .
    \end{equation}
\end{itemize}
After a number of iterations equal to the maximum desired depth of the
trellis diagram, $K_d$ we select
\begin{equation}
  {\vsigma_j}_{opt} = \argmin_{\vsigma_j} \Gamma(\vec{s}_{K_d-1} = \vec{\sigma}_j)
\end{equation}
and follow the survivor sequence $\mathcal{L}(\vec{s}_{K_d-1} =
{\vsigma_j}_{opt})$ backward until we reach the first state, which is
the most likely to be the actual state of the system at $K_d-1$
samples in the past. Finally we can get the detected symbol
$\hat{a}_{k}$ from the first component of the state, which is shifted
of $K_d-1$ with respect to the original symbol sequence $a_k$ because
the algorithm waits for the trellis diagram to be full before
detecting the first symbol. There is also a delay of $D$ samples
introduced by the filter $c$.

After trying some values we pick a trellis depth equal to ten times
the length of the state: $K_d = 20$. Since the input signal is a lot
longer than $K_d$, when the trellis diagram is full and we get a new
sample $\rho_k$, we delete the oldest values from the path metrics and
the survivor sequences. Since the path metrics are non-decreasing at a
certain point they will overflow, so we shift them back every 5000
samples by subtracting the smallest value from all the others.

\section{Max-Log-MAP}
In the Max-Log-Map receiver we use the same receiver structure and the
same parameters $t_0 = 33\frac{T}{4}$, $M_1 = 3$, $M_2 = 2$, $D=1$ of
section~\ref{sec:viterbi} so the overall system response $\psi(kT)$ is
the same one of Fig.~\ref{plot:psi_dfe} and we set the same state
length used in the previous case: $L_1 = 0$, $L_2 = M_2 = 2$.

The state used in this algorithm is
\begin{equation}
  \vec{s}_k = (a_k,a_{k-1}) \in \mathcal{S} = \{ \vec{\sigma}_1, \vec{\sigma}_2, \dots \vec{\sigma}_{16} \}
\end{equation}
where the possible states $\vsigma_i$ are the permutations of two
symbols taken from the alphabet $\mathcal{A}$.

Like in the Viterbi case we pre-compute the possible transitions
$\vsigma_i \rightarrow \vsigma_j$ and the values of $u_k =
f(\vsigma_j, \vsigma_i)$. We don't pre-compute the channel transition
metrics $c_k(j | i)$ because the algorithm ran at an acceptable speed
computing them when needed.

We process an input signal with the FBA algorithm by splitting it into
smaller chunks of $K_{in} = 600$ samples, which are overlapped by
$2(K_d-1)$ samples. We then run the algorithm on these chunks and
discard the first and the last $K_d-1$ samples from each output symbol
sequence and join them together to have the final output.  In this way
we limit the amount of memory needed to process a long signal and, by
discarding the first and last output symbols, we use the same number
of transitions to detect the symbol at every sample.  For the trellis
depth we pick the same value of the Viterbi case $K_d = 20$.

The forward procedure is very similar to the Viterbi algorithm: after
we have initialized the values of the logarithm of the forward metric
\begin{equation}
  \tilde{f}_{-1}(j) = 0 \qquad j=1,2,\dots 16
\end{equation}
since we don't know the initial state of the system, at each iteration
and for each state $j$
\begin{itemize}
  \item we compute the transition metrics
    \begin{equation}
      c_k(j|l) = -\abs{\rho_k - u_k}^2
    \end{equation}
    for every $l$ such that there is a transition $\vsigma_l
    \rightarrow \vsigma_j$,
  \item we compute the next possible values of $\tilde{f}_k(j)$
    \begin{equation}
      \tilde{f}_{k-1}(l) + c_k(j|l) \qquad \forall l : \vsigma_l
      \rightarrow \vsigma_j ,
    \end{equation}
  \item we pick the maximum of these metrics
    \begin{equation}
       \tilde{f}_{k}(j)  = \max_{l} \left[ \tilde{f}_{k-1}(l) + c_k(j|l) \right] 
    \end{equation}
    and save it as the forward metric of state $j$ at time $k$. This
    time we do not need to store the survivor sequence.
\end{itemize}
We stop the algorithm after $K_{in} - K_d$ steps because the rest will
not be used and we store the last forward metrics $\tilde{f}_{K_{in} -
  K_d}(j)$ because they will be the initial metrics for the following
chunk of samples.

We then apply a similar procedure to compute the logarithm of the
backward metrics $\tilde{b}_k(i)$. In this case we can not reuse the
metrics for the next chunk of samples because they depend on the last
received sample, so we always start from
\begin{equation}
  \tilde{b}_{K_{in}}(i) = 0 \qquad i = 1,2,\dots16x .
\end{equation}
Then:
\begin{itemize}
\item we compute the transition metrics
  \begin{equation}
    c_{k+1}(m|i) = -\abs{\rho_{k+1} - u_{k+1}}^2
  \end{equation}
  for every $m$ such that there is a transition $\vsigma_i \rightarrow
  \vsigma_m$,
  \item we compute the possible values of the backward metric at the
    previous instant $\tilde{b}_k(i)$
    \begin{equation}
      \tilde{b}_{k+1}(m) + c_{k+1}(m|i) \qquad \forall m : \vsigma_i
      \rightarrow \vsigma_m ,
    \end{equation}
    \item we pick the maximum of these metrics
    \begin{equation}
       \tilde{b}_{k}(i)  = \max_{m} \left[ \tilde{b}_{k+1}(m) + c_{k+1}(m|i) \right] 
    \end{equation}
    and save it as the backward metric of state $i$ at time $k$.
\end{itemize}

When we have both the forward and backward metrics we can sum them to
get the logarithm of the state metric
\begin{equation}
  \tilde{v}_k(i) = \tilde{f}_k(i) + \tilde{b}_k(i) \qquad i=1,2,\dots16 ,
\end{equation}
apply the Max-Log-Map approximation to compute the log-likelihood and
decide on the received symbols
\begin{align}
  \tilde{l}_k(\beta) &= \max_{\substack{i=1,2\dots16 \\ [\vsigma_i]_1 = \beta}} \tilde{v}_k(i) \\
  \hat{a}_k &= \argmax_{\beta \in \mathcal{A}} \tilde{l}_k(\beta)
\end{align}
Like in the Viterbi algorithm the output sequence is shifted of $K_d
-1$ and there is the delay $D$ introduced by the filter $c$.

\section{Symbol error comparison}
In Fig.~\ref{plot:pe_all} we plot the symbol errors obtained by the
simulation of the six receivers with $\Gamma$ between 8~and~14 dB
along with the theoretical lower bound on their values, given by
\begin{equation}
  P_e \geq 2 Q \left(\sqrt{\Gamma}\right) ,
\end{equation}
and the results obtained by simulating the same transmission system
with a channel with no ISI and additive white noise
\begin{align}
  y_k &= a_k + w_k \\
  w_k & \distr \mathcal{CN}(0, \sigma^2_w) \\
  \sigma^2_w &= \frac{\sigma^2_a}{\Gamma} \\
  \sigma^2_a &= 2 
\end{align}
that uses a threshold detector.

\begin{figure}[htbp]
  \centering
  \includegraphics[width=\textwidth]{matlab/pe_plot}
  \caption{Comparison of the symbol error with different receivers}
  \label{plot:pe_all}
\end{figure}
All the simulations ran with a uniformly distributed symbol sequence
of length 300000, in order to have a 95\% confidence interval of
$[0.5P_e, 2P_e]$ in the cases with the lowest $Pe = 10^{-4}$.

In the plots we can see that the two best performing algorithms are
Viterbi and FBA and their $P_e$ is very close to the simulated and the
theoretical bounds. Around 11 dB there are points that lie
below these limits, but this is due to the variance of the estimate
and if we ran the simulation using longer input sequences, these
points would return above the bounds. Since the two algorithms give
values of the $P_e$ very close to each other but the FBA requires
roughly twice the memory and execution time, the Viterbi algorithm
would be preferable.

The next best receivers are those with the matched filter, where the
DFE gives slightly better performances. They have the advantage of
being very fast, compared to the Viterbi and Max-Log-MAP algorithms.

The receivers with the anti-aliasing filters are the worst performing
ones, but this is expected since in the other receiver schemes we use
a filter that is matched exactly to the response of the channel while
here the DFE has to assume this role as well.
\end{document}

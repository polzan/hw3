\documentclass[a4paper,oneside]{article}
\usepackage[T1]{fontenc}
\usepackage[utf8]{inputenc}
\usepackage[english]{babel}

\usepackage[margin=2.54cm]{geometry}
\usepackage{amsmath}
\usepackage{siunitx}
\usepackage{listings}
\usepackage{color}
\usepackage{textcomp}
\usepackage{graphicx}
\usepackage{xr}
\usepackage{subcaption}
%\usepackage{changepage}
\usepackage[section]{placeins}
%\usepackage{hyperref}

%\strictpagecheck
\externaldocument{hw3_code}

\definecolor{matlabgreen}{RGB}{28,172,0}
\definecolor{matlablilas}{RGB}{170,55,241}

\newcommand{\includecode}[1]{\lstinputlisting[caption={\ttfamily #1.m},label={lst:#1}]{matlab/#1.m}}
\newcommand{\inlinecode}[1]{\lstinline[basicstyle=\ttfamily,keywordstyle={},stringstyle={},commentstyle={\itshape}]{#1}}

\renewcommand{\vec}[1]{\underline{#1}}
\renewcommand{\Re}[1]{\operatorname{Re}\left[#1\right]}
\newcommand{\E}[1]{\operatorname{E}\left[#1\right]}
\newcommand{\norm}[1]{\left\lVert#1\right\rVert}
\newcommand{\abs}[1]{\left|#1\right|}
\newcommand{\F}[1]{\operatorname{\mathcal{F}}\left[#1\right]}
\newcommand{\ceil}[1]{\left\lceil#1\right\rceil}
\newcommand{\floor}[1]{\left\lfloor#1\right\rfloor}
\newcommand{\Prob}[1]{\operatorname{P}\left[#1\right]}
\newcommand{\ProbC}[2]{\operatorname{P}\left[#1\middle|#2\right]}
\newcommand{\ind}[1]{\operatorname{\mathbbm{1}}\left\{#1\right\}}
\newcommand{\distr}[0]{\sim}
\newcommand{\unif}[1]{\mathcal{U}_{#1}}

\author{Enrico Polo \and Riccardo Zanol}
\title{Homework 3}

\begin{document}
\lstset{
  language=Matlab,
  basicstyle={\ttfamily \footnotesize},
  breaklines=true,
  morekeywords={true,false,warning,xlim,ylim},
  keywordstyle=\color{blue},
  stringstyle=\color{matlablilas},
  commentstyle={\color{matlabgreen} \itshape},
  numberstyle={\ttfamily \tiny},
  frame=leftline,
  showstringspaces=false,
  numbers=left,
  upquote=true,
}
\maketitle
\section*{Transmitter}
In the transmitter we generate the sequence of symbols $a_k$ by
producing a uniformly distributed sequnce of bits twice as long and
mapping each pair of bits to a symbol of the QPSK constellation $
\mathcal{A} = \{(1+j),(1-j),(-1-j),(-1+j)\}$. Then we upsample $a_k$
by a factor of 4 and filter it using the transfer function
\begin{equation}
  Q_c(z) = \frac{\beta z^{-10}}{1 - \alpha z^{-1}}
\end{equation}
that models the {\color{red} combination of the modulator filter and
  the channel}.  Since we will need it to have a finite length,
because the matched filter would otherwise not be causal, we truncate
its impulse response at $n=34$ (when $n > 34$, $q_c(nT/4) \leq
5\cdot10^{-5}$). In Fig.~\ref{plot:qc} we plot $q_c$ and in
Fig.~\ref{plot:Qf} there is the corresponding frequency response.
\begin{figure}[htbp]
  \centering
  \includegraphics[width=0.7\textwidth]{matlab/plot_qc}
  \caption{Impulse response of the combination of the modulator and
    the channel}
  \label{plot:qc}
\end{figure}
\begin{figure}[htbp]
  \centering
  \includegraphics[width=0.7\textwidth]{matlab/plot_Qf}
  \caption{Frequency response of the combination of the modulator and
    the channel}
  \label{plot:Qf}
\end{figure}

We then add {\color{red} the channel noise} $w_c(nT/4)$, assumed to be
a complex gaussian with power spectral density $\mathcal{P}_{w_c}(f) =
N_0$ in the band of the signal $S_c(nT/4)$. The noise power and PSD
can be obtained from the SNR:
\begin{align}
  \Gamma &= \frac{\sigma^2_a E_{q_c}}{\sigma^2_{w_c}} \\
  N_0 &= \sigma^2_{w_c}\frac{T}{4}
\end{align}
where $E_{q_c}$ is the energy of the filter $q_c$ and the power of the
symbol sequence is $\sigma^2_a = 2$.

The signal that gets to each one of the following receivers is
$r_c(nT/4) = S_c(nT/4) + w_c(nT/4)$.
\section*{Viterbi}

\section*{MaxLogMAP}

\end{document}
